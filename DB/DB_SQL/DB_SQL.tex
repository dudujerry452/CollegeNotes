\documentclass{article}        % 您的输入文件必须包含这两行
\usepackage{ctex}              % 添加对中文的支持
\begin{document}               % 以及文档末尾的 \end{document} 命令。
 
\section{SQL命令}

\subsection{查询结构}

基本结构

\begin{verbatim}
select 列名
    from 表名
    where 条件表达式
\end{verbatim}

产生一个关系作为结果,其中from中引用的表之间使用笛卡尔积连接

示例:给出所有女同学的姓名:

\begin{verbatim}
    select SNAME from S where SSEX = '女'
\end{verbatim}

详细结构
\begin{verbatim}
select [ALL | DISTINCT] <列名>
[, <列名> ] ...
from <表名> [, <表名> ] ...
[where <行条件表达式>]
[group by <列名> [, <列名> ] ...]
[having <条件表达式>]
[order by <列名> [asc | desc] ...]
\end{verbatim}

工作过程:

1.读取from子句的表或视图做笛卡尔积

2.where子句找出满足条件表达式的元组

3.按group by子句指定列名分组,值相等的元素为一组,
每个组产生结果表中的一条记录。可在没每组中与聚集函数;
若group子句带having,则只有满足条件的组才被输出

4.按select子句中给出的列名或表达式求值输出

5.order by子句对输出结果排序

\subsubsection{其他运算符}

where: 

- <, <=, =, <>或!=

- and, or, not

集合:

- union, intersect, except

集合成员资格:

- in, not in

谓词:

- exists, not exists, all, some, unique, between ... and ..., not between ... and ..., 
like, not like, is null ... 

聚集函数:

- avg, min ,max, sum, count ... 

\subsubsection{重复函数的处理}

语法:默认为保留重复元组,也可以用all指定。
若要去掉重复元组,可用distinct指明。

\subsubsection{from子句}

from子句列出查询需要访问的关系列表

当目标属性取自多个关系时,不混淆则可以不用显式指明来自哪个关系

\subsubsection{自然连接}

- from table1 natural join table2 

自然连接将参与连接的表相同的列名合并,其余属性跟随相同的列排布 


\subsection{分组聚集}

分组命令

\begin{verbatim}
group by 列名 [having 条件表达式]:
\end{verbatim}

按照列的值分组,对每个组别进行聚集运算

\begin{verbatim}
having
\end{verbatim}

则对分组进行选择


示例1:

\begin{verbatim}
select DNO, max(SAL), min(SAL)
from PROF 
group by DNO
\end{verbatim}

根据部门编号分组,对每个部门,求出最大工资和最小工资

示例2:

列出2009年讲授的每个课程段,至少有两名学生选课的总学分平均值

\begin{verbatim}
select course_id, sec_id, semeseter, year, avg(tot_cred)
    from takes natural join student
    where year = 2009
    group by course_id, sec_id, semeseter, year
    having count(ID) >= 2
\end{verbatim}

\subsection{嵌套子查询}

\subsubsection{集合成员资格}

\textbf{连接词in测试元组是否是集合中的成员}

\begin{verbatim}
select *
from S 
where SNAME in('张红', '王军')
\end{verbatim}

同理,使用not in测试元组不在集合中

\textbf{in子查询}

判断表达式的值是不是在子查询的结果集合中

例如

\begin{verbatim}
select SNO,SNAME
from S
where SNO in 
        (select SNO
        from SC 
        where CNO = '001'
        )
\end{verbatim}
选修了001号课程的学生的学号及姓名

再例如筛选同一张表的多个条件,此时用其他方法十分麻烦

\begin{verbatim}
select SNO,SNAME
from S
where CNO = '001'
and SNO in 
        (select SNO
        from SC 
        where CNO = '002'
        )
\end{verbatim}
选修了001号且选了002号课程的学生的学号及姓名

也可以与集合成员资格判断一起使用

\begin{verbatim}
select count(distinct SNO)
from takes
where(coutse_id, sec_id, semester, year) in
        (select coutse_id, sec_id, semester, year
        from teaches
        where ID = '10101'
        )
\end{verbatim}

此代码可以找出选修了10101号老师的课的不同的学生总数。

\subsubsection{集合的比较}

结构:

- 表达式 比较运算符 $\theta$ some(子查询)

意思是,表达式的值至少与子查询结果中的一个值相比满足比较运算符$\theta$

注意, =some 等价于 in, 但 !=some 不等价于 not in

- 表达式 比较运算符 $\theta$ all(子查询)

相似地,表达式的值与子查询结果中的所有值相比都满足比较运算符$\theta$

注意, !=all 等价于 not in, 但 =all 不等价于 in

\textbf{注意,集合的比较在可读性上比笛卡尔积优越,如如下例子}

\begin{verbatim} 
select distinct T.name 
from instructor T, instructor S
where T.salary > S.salary
and S.dept_name = 'Biology'
\end{verbatim}

可以改写为

\begin{verbatim}
select distinct name 
from instructor T 
where salary > some(
                select salary 
                from instructor S
                where S.dept_name = 'Biology'
                )
\end{verbatim}

选出工资大于生物系某个老师工资的老师的名字; \\

也可以进一步对聚集运算的结果操作:

\begin{verbatim} 
select SNO 
from SC 
group by SNO 
having avg(GRADE) >= all 
            (select avg(GRADE) 
            from SC 
            group by SNO)
\end{verbatim}

选出平均成绩大于所有学生平均成绩的学生学号

\subsubsection{空关系测试}

- [not] exists (子查询)

判段子查询的结果集合中是否有任何元组存在

注意:

\textbf{
in后的子查询与外层查询无关,每个子查询执行一次,
而exists后的子查询与外层查询有关,
使用了来自外层的相关查询名称,会执行多次
}

并且,exists子查询的名称只能来源于子查询和其父查询

例如,

- 列出2009年秋季和2010春季同时开课的所有课程id

\begin{verbatim}
select distinct course_id 
from section 
where semester = 'Fall' and year = 2009 and 
    course_id in ( select course_id
                from section
                where semester = 'Spring' and year = 2010
                );
\end{verbatim} 

可以改写为

\begin{verbatim}
select distinct course_id 
from section S1
where semester = 'Fall' and year = 2009 and 
    exists (select *
            from section S2
            where semester = 'Spring' and year = 2010
            and S1.course_id = S2.course_id
            );
\end{verbatim} 

\textbf{集合包含测试}

- 若关系A包含关系B(A为B的超集), 则 

    not exists (B [except/minus] A)  

- 对B except A 

    可以表达在B中存在, 在A中不存在的记录

    若B是A的子集,则B exceot A结果为空集 则not exists (B except A) 为真

例子:

列出选修了Biology系开设的所有课程的学生id,姓名 

\begin{verbatim} 
select S.ID, S.name 
from student S  
where not exists  
    (
        (select course_id 
        from course 
        where dept_name = 'Biology' )
        except 
        (
        select T.course_id 
        from takes as T 
        where S.ID = T.ID
        )
    )
\end{verbatim} 

思路:对于生物系开设的所有课程,对于每个学生,减去每个学生修的课,
如果结果为空,说明生物系的所有课是该学生的课的子集,即该学生选修了生物系的所有课\\ 

列出选修了所有课程的学生名

\begin{verbatim}  
select SNAME from S 
where not exists ( 
    select CNO from course 
    where not exists( 
        select * from SC 
        where SC.SNO = S.SNO and SC.CNO = course.CNO
    )
)
\end{verbatim}

思路:对于每一个学生,检查所有课程,若不存在这样一门课——不属于该学生选课的课程不存在,
那么该学生选修了所有课程




\subsubsection{重复元素存在性检测}

- [not] unique (子查询) 

若子查询中没有重复的元素(不存在的元素也算没有重复)则为真 \\

例子:

列出2017年最多开设一次的所有课程

\begin{verbatim} 
select T.course_id 
from course as T 
where unique 
    (select R.course_id 
    from section as R 
    where T.course_id = R.course_id 
    and R.year = 2017 )
\end{verbatim}

思路,对于每一个课程,检查该课程号在学期表中是否唯一,
值的注意的是一次都没开unique也算为真

等价代码:

\begin{verbatim} 
select T.course_id  
from course as T 
where 1 >= (
    select count(R.course_id) 
    from section as R 
    where T.course_id = R.course_id and 
    R.year = 2017
)
\end{verbatim}

\subsubsection{from子句中的子查询}

- (子查询) as 关系名 [列名, 列名 ...] 

将子查询的结果作为新的表运算\\ 

示例:找出平均成绩及格的学生

思路: 

(1)求出每个学生平均成绩 

(2)再从中找出及格的学生(替换having子句)

\textbf{(1)求出每个学生平均成绩的子查询表达}
\begin{verbatim} 
select SNAME,avg(GRADE) 
from S, SC 
where S.SNO = SC.SNO 
group by SNAME 
\end{verbatim}

\textbf{(2) 找出及格的学生}

\begin{verbatim} 
select SNAME, AVG_GRADE 
from ( 
    select SNAME, avg(GRADE) 
    from S, SC 
    where SC.SNO  = S.SNO 
    group by SC.SNO, SNAME 
) as result(SNAME, AVG_GRADE) 
where AVG_GRADE >= 60; 
\end{verbatim}

\subsection{with子句}

with子句提供定义临时关系的方法。

定义仅对包含with子句的查询有效

- 例子:找出最大预算值的系 

\begin{verbatim} 
with max_budget(value) as 
    (select max(budget) 
    from department) 
select dept_name, budget 
from department, max_budget 
where department.budget = max_budget.value;
\end{verbatim}

思路:先准备部门的最大预算,再筛选出这个最大预算值的部门 \\ 

例子:

查出所有工资总和大与所有系工资总额平均值的系

\begin{verbatim}  
with dept_total (dept_name, value) as 
    (select dept_name, sum(salary)
    from instructor 
    group by dept_name), 

    dept_total_avg(value) as 
    (select avg(value)
    from dept_total)

select dept_name 
from dept_total, dept_total_avg 
where dept_total.value >= dept_total_avg.value;
\end{verbatim}

思路:先求出每个系的工资总额,再求出所有系的工资总额的平均值,最后筛选出工资总额大于平均值的系
我认为,这样的写法十分优雅,而且很有条理,可读性好。

\subsubsection{标量子查询}

- 值返回单个属性的子查询,可出现在任意要求单个值的表达式中,
如select。 

例如,

\begin{verbatim}
select dept_name, 
    (select count(*)
    from instructor 
    where department.dept_name = instructor.dept_name) 
    as num_instructors 
from department;
\end{verbatim}

思路:对于每一个部门,先计算出教师总数,再直接输出

\subsection{插入操作}

插入操作可以使用VALUES,也可以使用子查询

\begin{verbatim}
    insert into [table name] 
        values([值, ] ...)
\end{verbatim}

\begin{verbatim}
    insert into [table name] ([列名, ] ...) 
        values([值, ] ...)
\end{verbatim}

\subsection{删除操作}

删除PROF中的所有元组

\begin{verbatim}
    delete from PROF
\end{verbatim}

删除王明老师所有的任课记录

\begin{verbatim}
    delete from PC 
    where PNO in(
        select PNO from PROF 
        where NAME = '王明'
    )
\end{verbatim}

\subsection{更新操作}

所有老师工资上调5\% 
\begin{verbatim}
    update instructor 
    set salary = salary*1.05
\end{verbatim}

将系主任的工资改为系平均工资 
\begin{verbatim}
    update PROF 
    set SAL = (select acg(SAL) 
                from PROF 
                where DNO = D01)
    where PNO(select DEAN 
                from DEPT 
                where DNO = D01)
\end{verbatim}

对工资超过100000的涨3\%,其余的涨5\%的工资
\begin{verbatim}
    update instructor 
    set salary = salary * 1.03 
    where salary > 100000

    update instructor 
    set salary = salary * 1.05
    where salary <= 100000
\end{verbatim}

注意命令顺序不能颠倒

\subsubsection{case更新操作}

语法格式:

\begin{verbatim}
    case 
        when pred1 then result1 
        when pred2 then result2 
        ... 
        when predn then resultn
        else result0
    end
\end{verbatim}

上一个示例可以改为:
\begin{verbatim}
    update instructor 
    set salary = case 
            when salary > 100000 then salary = salary * 1.03 
            else salary = salary * 1.05
\end{verbatim}

把student中每个tot\_cred属性设为学生成功学完所有课程成绩既不是F也不是空的学分总和

\begin{verbatim}
    update student S 
    set tot_cred = (
        select sum(credits)
        from takes natural join course 
        where S.ID = takes.ID 
        and takes.grade <> 'F' 
        and takes.grade is not null
    )   
\end{verbatim}

添加要求:没学完任何课程的学生学分为0
\begin{verbatim}
    update student S 
    set tot_cred = (
        select case
        when sum(credits) is not null then sum(credit) 
        else 0
        from takes natural join course 
        where S.ID = takes.ID 
        and takes.grade <> 'F' 
        and takes.grade is not null
    )
\end{verbatim}

\subsection{连接表达式}

自然连接:

结果的公共属性上取值相等,且只出现一次

join ... using (A1, A2 ... )

结果的公共属性子集(A1, A2 ... )取值相等,且只出现一次 

join ... on <谓词> 

结果的公共属性满足谓词的要求,且出现两次

\subsubsection{内连接和外连接}

内连接:只返回参与运算的两个表的匹配的部分

外连接:在内连接分基础上,加上左侧或右侧不匹配的部分(匹配内容为空的)

左外连接(left outer join) / 右外连接(right outer join) : 如上 

例如:

与该表连接

\begin{table}[htbp]
    \begin{minipage}{0.49\linewidth}
        \centering 
        \label{example1}
        \begin{tabular}{|c|c|}
            \hline 
            id&name \\
            \hline 
            1&Bob \\ 
            \hline 
            2&Alice \\ 
            \hline 
            3&King \\
            \hline
        \end{tabular}
    \end{minipage}
    \hfill
    \begin{minipage}{0.49\linewidth}
        \centering 
        \label{example2}
        \begin{tabular}{|c|c|}
            \hline 
            name&age  \\
            \hline 
            Bob & 22  \\ 
            \hline 
            Alice & 23  \\ 
            \hline 
            Jerry & 24\\
            \hline
        \end{tabular}
    \end{minipage}
\end{table}

内连接的效果为:

\begin{table}[htbp]
    \centering
    \label{example01}
    \begin{tabular}{|c|c|c|}
        \hline 
        id&name&age \\ 
        \hline 
        1&Bob&22 \\
        \hline 
        2&Alice&23 \\ 
        \hline 
    \end{tabular}
\end{table}

左外连接的效果为

\begin{table}[htbp]
    \centering
    \label{example02}
    \begin{tabular}{|c|c|c|}
        \hline 
        id&name&age \\ 
        \hline 
        1&Bob&22 \\
        \hline 
        2&Alice&23 \\ 
        \hline 
        3&King&null \\ 
        \hline 
    \end{tabular}
\end{table}

右外连接的效果为

\begin{table}[htbp]
    \centering
    \label{example03}
    \begin{tabular}{|c|c|c|}
        \hline 
        id&name&age \\ 
        \hline 
        1&Bob&22 \\
        \hline 
        2&Alice&23 \\ 
        \hline 
        null&Jerry&24 \\ 
        \hline 
    \end{tabular}
\end{table}

全外连接的效果为

\begin{table}[htbp]
    \centering
    \label{example04}
    \caption{全外连接}
    \begin{tabular}{|c|c|c|}
        \hline 
        id&name&age \\ 
        \hline 
        1&Bob&22 \\
        \hline 
        2&Alice&23 \\ 
        \hline 
        3&King&null \\ 
        \hline 
        null&Jerry&24 \\ 
        \hline 
    \end{tabular}
\end{table}

\subsection{视图}

作为虚关系,对用户可见的关系

定义:

\begin{verbatim}
    create view view_name[(列名[,列名]....)]
        as (查询表达式)
\end{verbatim}

视图的属性名缺省为子查询结果中的属性名,可以显示指明

删除视图

\begin{verbatim}
    drop view view_name
\end{verbatim}

视图在逻辑分层中的位置

\begin{table}[htbp]
    \centering
    \label{example001}
    \caption{视图的逻辑分层}
    \begin{tabular}{|c|c|}
        \hline 
        外模式&view\\ 
        \hline 
        模式&logic \\ 
        \hline 
        内模式&physics \\ 
        \hline 
    \end{tabular}
\end{table}

\subsubsection{物化视图}

某些DBS允许存储视图,但保证如果定义视图的实际关系改变,视图也跟着改变。

\subsubsection{视图更新}

称视图是可更新的,当且仅当

- from子句中仅有一个数据库关系

- select只包含关系属性名,不包含表达式,聚集以及distinct声明

- 出现在select子句中的属性限制null

- 查询中不包含group, having子句(保证单射)

\textbf{with check option} 

视图定义时,指定with check option,检查通过视图修改时,结果必须在视图中,
否则拒绝更新

在试图进行不符合view条件的更新时:

- 没有with check option :
可以更新,更新后元组不再出现在视图中

- 有with check option :
不可以更新

\subsection{事务}

开端:执行一条新的sql语句

结束:

- commit work 结束事务并提交 

- rollback work 事务没有完成,回滚数据

-  commmit/rollback 事务完成,其后的sql属于新的事务

\subsection{完整性约束}

- 完整性:

数据的正确性和相容性

- 完整性检查:

DMBS提供一种机制确保DB中的数据满足完整性规范\\

单个关系的约束 

\begin{verbatim}
    create table [约束]
\end{verbatim}

约束:

- not null 

- unique

-check <谓词> 

\subsubsection{属性值上的约束}

- not null:非空约束 要求某属性取值不能为空 

- unique:唯一性约束,要求该列为一超码,但隐式允许属性为空,除非显式声明

- check(P):要求每个值都满足check(P),

\textbf{check子句}

... TODO : 补完ppt

\subsection{参照完整性约束}

\subsubsection{references子句}

写法:

\begin{verbatim}
    dept_name varchar(20) references department 
\end{verbatim}

当违反约束时,通常直接拒绝执行造成破坏完整性的操作

\textbf{foreign key子句}

foreign key子句可以指明,如果参照关系上的删除或更新操作违反了约束,
系统需要在被参照关系中进行修改来满足参照完整性约束,而不是简单地拒绝操作

\subsubsection{修改基本关系主码的情况}

- RESTRICT 方式

只有当依赖关系中没有依赖基本关系的外码值,才可以修改/删除

- CASCADE 方式 

将依赖关系中依赖基本关系的外码值和主码一同修改/删除

- SETNULL 方式 

顾名思义,将依赖关系中依赖基本关系的外码值设为null

\subsubsection{约束的撤销和添加}

写法:

\begin{verbatim}
    撤销用: alter ... drop ... 
    添加用: alter ... add ... 
\end{verbatim}

示例 

\begin{verbatim}
    alter table S drop constraint S_PK 
    alter table SC add constraint SC_CHECK 
        check(Sno in select Sno from S)
\end{verbatim}

\textbf{域约束的创建添加与撤销}

\begin{verbatim}
    create domain AGE_DOMAIN smallint 
        constrint DC_AGE check (value <= 25 and value >= 15)
\end{verbatim}

\subsubsection{事务中对完整性约束的违反}

事务中的某一步会暂时违反完整性约束

例如

\begin{verbatim}
    create table person 
    (name varchar(20)
    spouse varchar(20)
    primary key name, 
    foreign key spouse references person):
        insert into table person( 'John', 'Mary')
\end{verbatim}

当我还没插入'Mary'时,不可能引用到'Mary',此时会暂时破坏完整性约束 

默认约束是立即检查,但延迟约束检查可以:

-  在约束声明中加入initially deferred子句,在事务结束时检查。
一个约束可以被指定为deferrable可延迟的约束,执行 set constraints constrain-list deferred 
命令作为事务的一部分,延迟到事务结束时检查

\subsection{断言}

语法:

\begin{verbatim}
    create assertion <断言名> check <条件> 
\end{verbatim}

断言是谓词,标大数据库总应该满足的条件。

- 一旦定义了断言,系统验证其有效性,
并且对每个可能违反该断言的更新操作都进行检查

- 这种检查会带来巨大的系统负载

- 对断言“所有X,P(X)“,是通过检查"not exists X, !P(X)" 实现的 \\ 

示例:

\begin{verbatim}
    create assertion ASSE2 check 
    (not exists 
    (select * from SC 
        where Cno in 
        (select Cno
            from C 
            where TEACHER = '张')
            and Sno in 
        (select Sno 
            from S 
            where SEX = 'M')))
\end{verbatim}

\subsection{(选看)SQL的数据类型与模式}

时间部分:

- time 一天中的时间,时分秒

- date 日历时期,年月日

- timestamp date和time的组合 

\subsubsection{日期和时间函数}

    extract (field from d);

获取当前日期,时间函数:

- current\_data 返回当前时间 

- current\_time 返回当前时间(带时区) 

- local\_time 返回当前本地时间(不带时区) 

- current\_timestamp, local\_timestamp 时间戳 

运算结果:interval区间类型

\subsubsection{类型转换和格式化函数}

- cast (e as t)

将表达式e转换为类型t 

示例:

\begin{verbatim}
    select cast(ID as nummeric(5)) as inst_id 
    from instructor 
    order by inst_id 
\end{verbatim}

- 格式化函数 

MYSQL: format 函数 
Oracle, PostgreSQL: to\_char, to\_number, to\_date 

\subsubsection{缺省值} 

语法:

\begin{verbatim}
    create table student {
        id varchar(5), 
        ... 
        tot_cred numeric(3,0) default 0, 
        ... 
    };
\end{verbatim}

\subsubsection{大对象}

字符数据的大对象clob, 二进制数据的blob 

语法 

\begin{verbatim}
    bookview clob (10KB) 
    image blob (10MB) 
    movie blob (2GB)
\end{verbatim}

\subsubsection{用户自定义类型}

\begin{verbatim}
    create type Dollars as numeric(12,2) final 
    create type Ponds as numeric(12,2) final 
\end{verbatim}

(一些数据库实现忽略final)

即便独特类型的定义是相同的,但不能相互赋值 

\subsubsection{create table的扩展}

- 创建于现有某模式相同的表 

\begin{verbatim}
    create table temp_instructor like instructor 
\end{verbatim}

- 创建包含查询的结果的表

\begin{verbatim} 
    create table t1 as (
        select 
        ...
    ) with data;
\end{verbatim}


\subsection{模式、目录和环境}

当代DB实现三层结构:

- 目录catalog 

- 模式schema 

- 关系table, 视图对象view \\ 

用户连接到数据库后,有一个默认的目录和模式,
默认目录和默认模式可以忽略 

- 用create schema 和drop schema创建、忽略模式 

\subsection{索引}

- 关系属性中的索引(Index), 可以提高在该属性上查询的速度,
无需扫描关系的全部内容.

- 索引是数据库物理模式的一部分,
而非逻辑部分。

- 索引对数据库的高效处理很重要。但由于索引的空间代价和
索引对更新处理的影响,要在维护何种索引上做出正确选择并不容易 \\ 

语法格式

\begin{verbatim} 
    create index <索引名> on <关系名>(<属性名>)
    create unique index <索引名> on <关系名>(<属性名>)

    drop index <索引名>
\end{verbatim}

\subsection{授权}

授权命令

\begin{verbatim}
    grant <权限名> on <表名|视图名> to 
    <用户/角色列表|public> [with grant option] 
\end{verbatim}

- 权限列表包括: selectm update, insert, delete , index 
alter drop, resource 以及它们的总和all, 其中对select update insert 可以指定属性列表 

- with grant option表示获得权限的用户可以把权限再次授予其他用户\\ 

例子:

\begin{verbatim}
    grant select, insert on S to Liming 
    with grant option 

    grant all on S to public 

    grant update(budget) on department to Amit, satoshi
\end{verbatim}

- 收回权限

\begin{verbatim}
    revoke 表级权限 on {表名|视图名} from {用户[用户,]... | public}
\end{verbatim}

收回权限的操作是递归的

\subsubsection{角色}

- 创建角色:

create role instructor; 

- 授权角色: 

grant select on takes to instructor;

- 角色可以授予给角色和其他角色 

\begin{verbatim}
grant instructor to Amit; 
create role dean; 
grant instructor to dean; 
grant dean to Satoshi;
\end{verbatim}

\subsubsection{授权图}

授权图是一棵树,节点是用户,当且仅当存在从根节点R到结点
A的路径时,A才拥有权限 

- 级联收回(默认) 

收回权限时,若该用户已将权限授予其他用户,那么也一并收回其子节点的权限,
但如果两个结点互为子节点,则循环递归,难以收回\\ 

- 防止级联收回 restrict 

\begin{verbatim}
    revoke select on department from Amit restrict; 
\end{verbatim}
如果存在任何级联收回,返回错误,不执行任何收权动作 

- 仅收回grant option,保留其他权限 

\begin{verbatim}
    revoke grant option for select on department from Amit; 
\end{verbatim}

一些数据库的实现不支持以上语法;可以收回权限,然后不带grant option重新授权 \\

- 通过角色授权, 避免用户授权的级联收回 

1)设置当前会话的当前角色为已定义角色 
set role role\_name 

2)有当前角色授权,授权语句如 
set role role\_name 

\subsubsection{数据库级权限授权}

语法 

\begin{verbatim}
    grant 数据库级权限 to {user[user ,]...|public }
\end{verbatim}

数据库级权限包括 

- connnect 允许用户在database语句中指定数据库 

- resource connect权限+建表,删除表及索引权利 

- dba resource权限+授予或撤销其他用户的connect, resource, dba权限 

不允许dba撤销自己的dba权限 

\subsubsection{行级授权}

\begin{verbatim}
    ID = sys_context('USERENV', 'SESSION_USER')
\end{verbatim}

系统将该谓词添加到使用takes关系的每个where子句中,每个 
学生只能看到与自己ID有关的数据 

\section{使用程序设计语言访问数据库}

动态SQL

用函数或者方法与数据库API交互 

- JDBC

- 从python访问数据库

- ODBC 

嵌入式SQL

- sql嵌入到宿主语言中 

- 通过预编译程序处理嵌入宿主语言的sql

\subsection{JDBC}

Java DataBase Connectivity ,面向java的数据库API,
原理,体系结构和用法基本都类似ODBC 

工作原理: 

\begin{table}[htbp] 
    \centering 
    \caption{JDBC的系统架构}

    \begin{tabular}{|c|c|}
        \hline 
        Java application&$\downarrow$\\
        \hline 
        JDBC API& $\downarrow$\\ 
        \hline 
        JDBC Driver Manager& $\downarrow$\\ 
        \hline 
        JDBC Driver& JDBC Driver \\ 
        \hline 
        Sql Server & Oracle \\
        \hline
    \end{tabular}

\section{触发器}

举例:职工工资增幅不超过10\% 

\begin{verbatim}
create trigger RAISE_LIMIT 
after update of SALon on EMP 
    referencing new row as nrowold row as orow 
    for each row 
    when (nrow.SAL > 1.1*orow.SAL)
        begin atomic 
            signal SQLSTATE('7500','Salary increase 10\%')
        end
\end{verbatim}
    

\end{table}







\end{document}     

