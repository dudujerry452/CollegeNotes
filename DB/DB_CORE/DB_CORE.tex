\documentclass{article}        % 您的输入文件必须包含这两行
\usepackage{ctex}              % 添加对中文的支持
\begin{document}           % 以及文档末尾的 \end{document} 命令。
 
\title{数据库内核}

\section{文件组织}

数据库是文件的集合,文件是记录的集合,记录是字段的集合 

数据库=\{文件\},文件=\{记录\},记录=\{字段\}

\subsection{定长记录}

\subsubsection{简单方法} 

记录i从字节n*(i-1)开始存储,其中n是记录大小,记录存取简单,
但记录可能跨块

- 删除记录 

\subsubsection{自由(空闲)链表}

在文件头中存储第一条被删记录的地址 

在第一条被删记录中存储第二条被删记录的地址 以此类推

这些存储的地址可视为指针,因为它们指向记录的位置 

\subsection{变长记录}

例如,TEXT,Book等长度不定的字段

实现:

- 使用分隔符

- 使用字符串存储

\subsubsection{变长记录的实现}

- 定长按照顺序存储

- 变长的属性用定长单元实现(offset, length)来表示 
实际属性存储在所有定长属性的后面

- 空值用空值位图(null-value bitmap)来表示 

\begin{table}[htbp]
  \centering
  \caption{变长数据的存储方式}
  \begin{tabular}{|c|c|c|c|c|c|c|c|}
    \hline 
    21,5&26,10&36,10&65000&0000&10101&Srinivasan&comp.Sci. \\
    \hline
    4 bytes&4 bytes&4 bytes& 4 bytes&null bitmap&5 byte& 10 byte& 10 byte \\
  \end{tabular}
\end{table}

\subsubsection{变长记录varchar与char*}

vaarchar和char在内存中的缓冲区里都是以完整的长度(即定义的最大长度)
来存放的,不足部分用填补字符来补充。仅仅是存放到磁盘上时,才会以压缩算法处理后存放 

存放到磁盘上时,对于varchar(n),可能有两种存放方式:

- 第一个byte(有可能是两个,即varchar最大长度超过是256)不存放实际数据,而是存放length。
接下来存放<=n个的实际字符

- 先存放<n个byte的实际字符,然后用一个不可显示的字符来标志该字段结束 

\subsubsection{变长记录的表示}

变长记录的表示有字节串形式和定长形式两种 

- 字节串形式 Bte string representation 

- 定长形式 Fixed length\\ 

1.尾标志法 

把每个记录看成是连续的字符串,然后再每个记录的尾部附加
“记录尾标志符“,标明记录结束 

2.记录长度法

在记录的开头添加该记录的长度 

\subsubsection{分槽的页结构}

每块的开始设置一个块首部 

Slotted page header contains: 文件头存储 

- number of record entries 记录个数 

- end of free space in the block 空闲空间尾部 

- location and size of each recored 每个记录维护和大小的数组

实际记录从块的末尾处开始在块中连续分配空间

\begin{table}[htbp]
  \centering
  \caption{分槽的单块中的结构}
  \begin{tabular}{|c|c|c|c|c|c|c|c|}
    \hline 
    文件头File header&...&....&---Free Space---&Record1&Recored2&...&Recordn(block end) \\
    \hline
  \end{tabular}
\end{table}

\subsubsection{变长的定长实现}

用大定长装作是变长,使用 

保留空间 Reserved space - 暂时不用的空着\\ 

但是有通过指针方法提高效率:

借用定长管理变长(?)

- 1 anchor lock 记录链上的第一个记录

- 2 overflow block 记录非第一个记录

\subsection{存储大对象}

数据库通常要存储比磁盘块大的数据 

- 如,blob或者clob

许多数据库内部限制一条记录的规模不能比一个块的规模更大

解决方案:大对象可以:

- 存储为文件系统中的文件 

- 存储为由数据库管理的文件 

- 分成小块在分离的关系中的不同的元组

TOAST超尺寸属性存储技术(烤面包?) 

\subsection{文件中记录的组织}

- 堆Heap文件组织  

一条记录可以存放在文件中的任何地方 只要哪个地方有空

记录没有顺序 插入策略:无锋堆后,有缝插针 \\ 

- 多标聚簇

\end{document}     



