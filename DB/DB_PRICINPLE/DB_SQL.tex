\documentclass{article}        % 您的输入文件必须包含这两行
\usepackage{ctex}              % 添加对中文的支持
\begin{document}           % 以及文档末尾的 \end{document} 命令。
 
\section{temp }

\subsection{关系代数}

过程化查询语言:\\ 

6种基本运算:

- 选择 $\sigma$ 

- 投影 $\Pi$

- 联合 $\bigcup$

- 集合差 $-$ 

- 笛卡尔积 $\times $

- 更名 $\rho$

\subsubsection{投影运算}

- 表达形式: $\Pi_{A_1,A_2,...,A_k}(r) $ 

其中$A_1,A_2$是属性名,r是关系名 

- 结果是只包含列表中属性的关系 

- 因为关系是集合, 重复的行均被去除 

\subsubsection{选择运算}

- 表达形式:
$ \sigma_p (r) $ ,
$p$是选择谓词 

- 定义:
$ \sigma_p (r) = \{t | t \in r \ and \  p(t) \} $

其中p可以是由and($\wedge$) or($\vee$) not($\neg$) 连接的多个谓词组成的命题表达式,
每一个谓词的表达形式为

<属性> op <属性> or <常量> 

op为比较运算符,包括$=, \neq, >, \le, <, \ge $

例如,$\sigma_{dept_name='Physics'}(instructor) $

\subsubsection{并运算}

表达形式: $r\cup s$

定义:$r\cup s = \{ t|t \in r \ or \  t\in s \} $ 

要使$r\cup s$有意义,则需要:

- r,s必须是同元的,即属性数目必须相同

- 属性的域必须是兼容的(如:r关系的第二列与s关系的第二列的值域类型相同)

例:找出2009年秋季或2010年春季的课程

$ \Pi_{course\_id}(\sigma_{semester="Fall" \wedge year=2009}(section)) \cup  
  \Pi_{course\_id}(\sigma_{semester="Spring" \wedge year=2010(section)}) $

\subsubsection{集合差运算}

表达形式:$r - s$ 

定义:$ r - s = \{ t | t \in \ and \ t \notin s \} $

集合差必须在相容的关系间运行

- r,s 必须是同元的(属性数目必须相同)

- 属性的域必须是兼容的

例:找出2009年秋季学期开,但不在2010春季学期开的课程

\subsubsection{交运算}

表达形式: $r \cap s$

定义: $r \cap s = \{ t|t \in r \ and \  t \in s \} $

集合的交运算必须在相容的关系间运行(同集合的差运算)

可转化为: $ r \cap s = r - (r-s) $

\subsubsection{笛卡尔积运算}

表达形式: $r \times s $

定义:$r \times s = \{ t\ q|t \in r \ and\ q \in s\} $

设关系r(R)与关系s(S)的属性是不同的,即($R \cup S = \emptyset$) 

否则,需要命名机制加以解决

\subsubsection{更名运算}

允许我们命名或重新指定代数表达式/关系的名字 

表达形式: $\rho_x(E) $返回E的名字x 

若E是n元的,则 

$ \rho_{x(A_1,A_2,...,A_n)}(E) $ 

\subsubsection{关系运算的组合}

关系运算的结果仍为关系,因此多个关系表达式可以组合为一个关系代数表达式 

例如,

$\Pi_{name}(\sigma_{dept_name='Physics'}(instructor)) $

\subsection{附加的关系代数运算}

\subsubsection{自然连接}

自然连接运算: $ r \bowtie s $

定义:
$ r\bowtie s = \Pi_{R\cup S}(\sigma_{r.A_1=s.A_1\wedge r.A_2=s.A_2 ... \wedge r.A_n=s.A_n}(r\times s))$

例如 $ \Pi_{name,course_id}(instructor\bowtie teaches)$ 

其中$R\cap S = {A_1,A_2,...,A_n}$, R, S, 表示属性的集合;
若$R\cap S = \emptyset$,则 $r\bowtie s = r \times s$

\subsubsection{赋值运算}

临时关系变量$ \leftarrow $ 关系代数表达式 

将右侧的表达式结果赋值给左侧的临时关系变量 

例如$ r \bowtie s$ 等价于 

$ temp1 \leftarrow r \times s $ 

$ temp2 \leftarrow \sigma_{r.A_1 \ wedge} $ 

// TODO 

\subsubsection{外连接}

外连接(outer-join)运算:对自然连接的扩充 

1)左外连接(符号:$\bowtie$左侧两端点加两条水平短线)

左侧表的元组全保留,左侧与右侧表不匹配的用null填充 

$(r\bowtie s)\cup (r-\Pi_R(r\bowtie s)\times(null,...,null))$,
其中常数关系$(null,...,null)$的模式是S-R

2)右外连接(符号同,但是方向相反) 

与左侧对应 

3)全外连接 

左侧右侧的元组全部保留,不匹配的用null填充

$(r\bowtie s)\cup (r-\Pi_R(r\bowtie s)\times(null,...,null)) \cup (s-\Pi_S(r\bowtie s))\times (null,...,null)$,

\subsubsection{广义投影}

- 表达形式:$\Pi_{F1,F2,...,Fn}(E) $

- Fn是涉及到常量和属性的算术表达式 

- 表达式as属性名 

例如,${Pi_{name,age+1 as virtual\_age}(student)}$

\subsubsection{聚集}

例如, 
$g_{avg(grade)}(选课)$ 
结果:avg(grade) = 86.5

例如,
求数学的最高成绩
$g_{max(grade)}(\sigma_{course\_name='math}(courses\bowtie takes)) $

- 分组聚集 

$_{group1, group2, ..., group k}g_{avg(grade)}(student)$ 

分别返回k个组中学生平均成绩

\section{元组关系演算}

- 元组关系演算(Tuple Relational Calculus) 

非过程化查询语言

- 表达形式

$ \{t | P(T) \} $ 使谓词P为真的元组t的集合 

例1: 找出所有工资在8000以上的教师的ID, name, dept\_name, salary 

$ \{ t| t\in instructor \wedge t[salary]>8000 \} $

\subsection{存在两次和蕴含结构}

- 谓词中的存在结构:$\exists$ 存在量词

- $\exists t \in r(Q(t)) \equiv 关系r中存在元组t使谓词Qt(t)为真 $ 

例2: 找出所有工资在8000以上的教师ID(单个属性) \\

$ \{\ t \ |\  \exists \  s \in instructor(t[ID]=s[ID] \wedge s[salary] > 8000)\ \} $

例3: 找出位置在Watson楼中系的所有教师姓名(涉及两个关系) 

$ \{t | s \in instructor(t[name]=s[name] \ \wedge \ \exists \ u \in department(u[dept\_ name] = s[dept\_name] \wedge u[building]='Watson'))\} $ \\ 

例4: 找出2009年秋季或者2010年春季学期,或者这两个学期都开设的所有课程的id集合 

$ \{ t| \exists s \in section(t[course\_id]=s[course\_id] \wedge s[semeseter]='Fall' \wedge s[year]=2009) 
\vee \exists u \in section(u[course\_id]=t[course\_id] \wedge u[semeseter]='Spring' \wedge u[year]=2010)\} $


- 谓词中的蕴含结构: $\Rightarrow$ 

$P \Rightarrow Q$, 即“如果P为真,则Q必为真“ 

真值表:

\begin{table}[htbp]
    \centering 
    \caption{蕴含的真值表}

    \begin{tabular}{|c|c|c|}
        \hline 
        P&Q&$P\rightarrow Q$\\ 
        \hline 
        T&T&T \\
        \hline
        T&T&F \\
        \hline
        F&T&T \\
        \hline
        F&F&T \\
        \hline
    \end{tabular}
\end{table} 

\subsection{所有结构}

- 谓词汇总的所有结构, $\forall$  

$ \forall t \in r(Q(t)) \equiv $对关系r中满足谓词Q(t)的所有元组 

例如,找出那些选了生物系全部课程的学生ID,涉及三个表 

$ t | \exists r \in student(r[ID] = t[ID] \wedge \forall u \in course(u[dept_name]='Biology) \Rightarrow \exists s \in takes(t[ID] = s[ID] \wedge s[course\_id]=u[course\_id])) $


\subsection{元组关系演算中的等价性}

- 等价性 

1.$P_1 \wedge P_2$ 等价于 $\neg(\neg(P_1)\vee \neg(P_2)) $ 

2.$\forall t \in r(P(t))$ 等价于 $\neg \exists t \in r(\neg P(t)) $  全称量词等价

3. $P_1 \rightarrow P_2$ 等价于 $\neg(P_1)\vee P_2 $ 蕴含等价 

\subsection{元组关系演算中的安全性}

任何可能会产生无限关系的表达式是不安全的 

- 公式P的域,计作dom(P)。P所引用的\textbf{所有值}的集合 

它包括在P中出现在P中的所有关系内的每个值以及提到的所有值的集合 

例如,

$ dom(t \in instructor \wedge t[salary]>8000) $ 

是包括8000和出现在instructor中的所有值的集合

$dom(\neg(t\in instructor)) $  是instructor的所有值

\section{域关系演算}

域关系演算使用从属性域中取值的域变量,是与元组关系演算能力等价的关系演算 

表达方法: ${<a_1,a_2,a_3,...,a_n>|w(o_1,o_2,..,o_k)}$其中w为谓词构成的集合





\end{document}     



